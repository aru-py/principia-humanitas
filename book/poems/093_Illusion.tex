\poem{Illusion}{Illusion = \frac{P \times (C + A - F)}{M \times Q}}{\item $P$: \index{Perceptibility}\textit{Perceptibility}. The ability to notice details within the environment. High perceptibility implies a capacity to spot subtleties that might confirm or negate an illusion.
\item $C$: \index{Culture}\textit{Culture}. Refers to how cultural background and beliefs impact an individual's susceptibility to illusions. Some cultures, particularly those with a strong emphasis on magic, increase the likelihood of perceiving illusions.
\item $A$: \index{Awareness}\textit{Awareness}. An individual's level of consciousness. Better awareness enables more critical analysis of sensory input, helping reduce illusion's impact.
\item $F$: \index{Fatigue}\textit{Fatigue}. Physical or mental exhaustion. Fatigue impairs information processing, making illusions seem more credible.
\item $M$: \index{Media}\textit{Media}. Exposure to illusion-using media like movies or magic shows. It can either dull or heighten illusion susceptibility.
\item $Q$: \index{Questioning}\textit{Questioning}. The act of critically examining perceptions and realities. Higher questioning diminishes illusion effects by promoting a more analytical view of sensory data.
}