\poem{Dreams}{Dreams = \left(\frac{I^n}{R + S} \right) \cdot e^{-\lambda t} + O}{\item $I$: \index{Imagination}\textit{Imagination}. The level of creative and imaginative thinking a person has before sleep. It drives the ability to visualize and mentally explore new scenarios, crucial for dreaming.
\item $R$: \index{Reality}\textit{Reality}. Quantifies the dream's connections with real-life experiences. A higher value indicates dreams closely tied to the dreamer's life, affecting dream content and engagement with reality.
\item $S$: \index{Stress}\textit{Stress}. The level of psychological tension before sleep. Stress can distort dream experiences, influencing their quality and vividness.
\item $\lambda$: \index{Decay}\textit{Decay}. A factor representing how sleep quality influences dream vividness over time. Better sleep yields more vivid dreams initially, but the effect declines during sleep.
\item $O$: \index{Baseline}\textit{Baseline}. Represents the universal level of dream content, independent of personalized factors such as stress or imagination. This is the core of dreaming, experienced by all.
}