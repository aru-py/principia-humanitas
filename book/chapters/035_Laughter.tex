\chapter{Laughter}

\begin{equation}
\text{Laughter} = \frac{C \cdot H \cdot S}{P + A} + \sqrt{\frac{R}{F + 1}}
\end{equation}

\textbf{Where:}

\begin{itemize}
    \item $C$: Contextual novelty or the uniqueness of the situation.
    \item $H$: The harmony or relatability of the laughter-inducing stimulus with the individual's experiences.
    \item $S$: Social bonding or the degree to which the situation facilitates connection with others.
    \item $P$: Personal barriers, including stress or inhibitions that reduce the likelihood of laughter.
    \item $A$: The absorption or distraction by external or internal factors unrelated to the humorous context.
    \item $R$: The resilience or mental flexibility of the individual, allowing them to appreciate or generate humor.
    \item $F$: The familiarity with the laughter-inducing stimulus, where higher familiarity might reduce the impact of humor.
\end{itemize}