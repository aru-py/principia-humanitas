\chapter{Revenge}

\begin{equation}
\text{Revenge} = \left( \frac{C \times (I + E)}{F + M} \right)^\alpha - L
\end{equation}

\textbf{Where:}

\begin{itemize}
    \item $C$: The depth of the initial harm or conflict.
    \item $I$: The intensity of the emotional response to the harm.
    \item $E$: The level of perceived enmity or opposition.
    \item $F$: The degree of forgiveness or willingness to move on.
    \item $M$: The presence of mitigating circumstances or alternative perspectives that reduce the desire for revenge.
    \item $\alpha$: The amplification factor, influenced by societal and cultural norms surrounding revenge.
    \item $L$: The potential loss (emotional, moral, social, or material) associated with seeking revenge.
\end{itemize}